%%%%%%%%%%%%%%%%%%%%%%%%%%%%%%%%%%%%%%%%%
% Thin Sectioned Essay
% LaTeX Template
% Version 1.0 (3/8/13)
%
% This template has been downloaded from:
% http://www.LaTeXTemplates.com
%
% Original Author:
% Nicolas Diaz (nsdiaz@uc.cl) with extensive modifications by:
% Vel (vel@latextemplates.com)
%
% License:
% CC BY-NC-SA 3.0 (http://creativecommons.org/licenses/by-nc-sa/3.0/)
%
%%%%%%%%%%%%%%%%%%%%%%%%%%%%%%%%%%%%%%%%%

%----------------------------------------------------------------------------------------
%	PACKAGES AND OTHER DOCUMENT CONFIGURATIONS
%----------------------------------------------------------------------------------------

\documentclass[a4paper, 12pt]{article} % Font size (can be 10pt, 11pt or 12pt) and paper size (remove a4paper for US letter paper)

\usepackage[protrusion=true,expansion=true]{microtype} % Better typography
\usepackage{graphicx} % Required for including pictures
\usepackage{wrapfig} % Allows in-line images
\usepackage{caption}
\usepackage{amsmath}
\usepackage{mathtools}
\usepackage{float}
\usepackage{mathpazo} % Use the Palatino font
\usepackage[T1]{fontenc} % Required for accented characters
\linespread{1.05} % Change line spacing here, Palatino benefits from a slight increase by default

\makeatletter
\renewcommand\@biblabel[1]{\textbf{#1.}} % Change the square brackets for each bibliography item from '[1]' to '1.'
\renewcommand{\@listI}{\itemsep=0pt} % Reduce the space between items in the itemize and enumerate environments and the bibliography

\renewcommand{\maketitle}{ % Customize the title - do not edit title and author name here, see the TITLE block below
\begin{flushright} % Right align
{\LARGE\@title} % Increase the font size of the title

\vspace{50pt} % Some vertical space between the title and author name

{\large\@author} % Author name
\\\@date % Date

\vspace{40pt} % Some vertical space between the author block and abstract
\end{flushright}
}

%----------------------------------------------------------------------------------------
%	TITLE
%----------------------------------------------------------------------------------------

\title{\textbf{Soluci\'on al \'atamo de Hidrogeno relativista}\\ % Title
} % Subtitle

\author{\textsc{Juan David Orjuela Zu\~niga \\ Juan Nicol\'as Garavito Camargo} % Author
\\{\textit{Universidad de los Andes, Bogot\'a, Colombia.}}} % Institution

\date{\today} % Date

%----------------------------------------------------------------------------------------

\begin{document}

\maketitle % Print the title section

%----------------------------------------------------------------------------------------
%	ABSTRACT AND KEYWORDS
%----------------------------------------------------------------------------------------

%\renewcommand{\abstractname}{Summary} % Uncomment to change the name of the abstract to something else

\begin{abstract}
abstract...
\end{abstract}
\hspace*{3,6mm}\textit{Keywords:}  % Keywords

\vspace{30pt} % Some vertical space between the abstract and first section

%----------------------------------------------------------------------------------------
%	ESSAY BODY
%----------------------------------------------------------------------------------------

\section*{Antecedentes historicos:}

\section*{Ecuaci\'on de Dirac:}

\section*{\'Atomo de Hidrogeno:}

La ecuaci\'on de Dirac para un potencial radial $V(r)$ es:

\begin{equation}\label{eq:diracvr}
(c \widetilde{\alpha} \cdot \hat{p} + \widetilde{\beta} m c^2 + V(r) ) \phi(r) = E \phi (r)
\end{equation}

En (\ref{eq:diracvr}) el termino que se debe tranformar a coordenadas radiales
 es $\widetilde{\alpha} \cdot \hat{\mathbf{p}}$, ya que \textbf{$\beta$} es independiente del 
sistema de coordenadas. 

\begin{equation}\label{alphap}
\widetilde{\alpha} \cdot \hat{\mathbf{p}} = -i \hbar \widetilde{\alpha} \cdot \nabla
\end{equation}

Haciendo uso de la identidad vectorial: 

\begin{equation}\label{eq:identity}
\nabla = \hat{\mathbf{r}}(\hat{\mathbf{r}}\cdot \nabla) - \hat{\mathbf{r}} \times (\hat{\mathbf{r}} \times \nabla)  
\end{equation}

Teniendo encuenta la simetr\'ia esferica del sistema, es decir donde $\partial/\partial \theta = 0$
y $\partial/\partial \phi = 0$  (\ref{eq:identity}) se puede expresar:

\begin{equation}\label{eq:nabla1}
\mathbf\nabla = \hat{\mathbf{r}}(\dfrac{\partial}{\partial r}) - \hat{\mathbf{r}} \times (\hat{\mathbf{r}} \times \nabla)
\end{equation}

El segundo termino en \ref{eq:nabla1} se puede expresar en terminos del momento angular $\mathbf{L} = -i\hbar \mathbf{r} \times \mathbf{\nabla}$.

\begin{equation}\label{eq:nabla}
\mathbf{\nabla} =  \hat{\mathbf{r}}(\dfrac{\partial}{\partial r}) - \dfrac{i}{\hbar}\dfrac{\hat{\mathbf{r}}}{r} \times \mathbf{L}
\end{equation}

Por lo tanto \ref{eq:alphap} se puede escribir en terminos de (\ref{eq:nabla}) como:

\begin{equation}\label{eq:alphap2}
\widetilde{\alpha}\cdot \mathbf{\hat{p}} = -i\hbar \widetilde({\alpha}\cdot \hat{\mathbf{r}}(\dfrac{\partial}{\partial r}) - 
\dfrac{i}{\hbar}\widetilde{\alpha}\cdot \dfrac{\hat{\mathbf{r}}}{r} \times \mathbf{L})
\end{equation}

El \'ultimo termino de (\ref{eq:alphap2}) se puede expresar como:

\begin{equation}
\widetilde{\alpha}\cdot\hat{\mathbf{r}}\widetilde{\alpha} \cdot \hat{\mathbf{L}} = i\widetilde{\sigma}\cdot \hat{\mathbf{r}} \times \mathbf{L}
\end{equation}

\begin{equation}
i\widetilde{\alpha}\cdot\hat{\mathbf{r}}\widetilde{\sigma} \cdot \hat{\mathbf{L}} = -\widetilde{\alpha}\cdot \hat{\mathbf{r}} \times \mathbf{L}
\end{equation}

\section*{Apendice: Demostraci\'ones de propiedades:}

\subsection*{$\widetilde{\gamma_5}$ properties:}

\[
\widetilde{\gamma_5} = 
\begin{pmatrix} 
0 & - \widetilde{I} \\
-\widetilde{I} & 0 \\
\end{pmatrix}
\]

Donde $\widetilde{I}$ es la matriz identidad de dos dimensiones. operando  $\widetilde{I}$ 
en $\widetilde{\alpha}$ y $\widetilde{\sigma}$ obtenemos:

\[
\widetilde{\gamma_5}\widetilde{\alpha_i} = 
\begin{pmatrix} 
0 & - \widetilde{I} \\
-\widetilde{I} & 0 \\
\end{pmatrix} 
\begin{pmatrix} 
0 & \sigma_i \\
\sigma_i & 0 \\
\end{pmatrix} 
= 
\begin{pmatrix}
-\sigma_i & 0 \\
0 & -\sigma_i
\end{pmatrix}
= -\widetilde{\sigma_i}
\]

y analogamente:

\[
\widetilde{\gamma_5}\widetilde{\sigma_i} = 
\begin{pmatrix} 
0 & - \widetilde{I} \\
-\widetilde{I} & 0 \\
\end{pmatrix} 
\begin{pmatrix} 
\sigma_i & 0 \\
0 & \sigma_i  \\
\end{pmatrix} 
= 
\begin{pmatrix}
0 & -\sigma_i \\
-\sigma_i & 0 \\
\end{pmatrix}
= -\widetilde{\alpha_i}
\]

%	BIBLIOGRAPHY
%----------------------------------------------------------------------------------------

\bibliographystyle{unsrt}

\bibliography{bibliografia}

%----------------------------------------------------------------------------------------

\end{document}
